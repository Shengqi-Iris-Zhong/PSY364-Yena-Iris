% Options for packages loaded elsewhere
\PassOptionsToPackage{unicode}{hyperref}
\PassOptionsToPackage{hyphens}{url}
%
\documentclass[
  english,
  man]{apa6}
\usepackage{lmodern}
\usepackage{amssymb,amsmath}
\usepackage{ifxetex,ifluatex}
\ifnum 0\ifxetex 1\fi\ifluatex 1\fi=0 % if pdftex
  \usepackage[T1]{fontenc}
  \usepackage[utf8]{inputenc}
  \usepackage{textcomp} % provide euro and other symbols
\else % if luatex or xetex
  \usepackage{unicode-math}
  \defaultfontfeatures{Scale=MatchLowercase}
  \defaultfontfeatures[\rmfamily]{Ligatures=TeX,Scale=1}
\fi
% Use upquote if available, for straight quotes in verbatim environments
\IfFileExists{upquote.sty}{\usepackage{upquote}}{}
\IfFileExists{microtype.sty}{% use microtype if available
  \usepackage[]{microtype}
  \UseMicrotypeSet[protrusion]{basicmath} % disable protrusion for tt fonts
}{}
\makeatletter
\@ifundefined{KOMAClassName}{% if non-KOMA class
  \IfFileExists{parskip.sty}{%
    \usepackage{parskip}
  }{% else
    \setlength{\parindent}{0pt}
    \setlength{\parskip}{6pt plus 2pt minus 1pt}}
}{% if KOMA class
  \KOMAoptions{parskip=half}}
\makeatother
\usepackage{xcolor}
\IfFileExists{xurl.sty}{\usepackage{xurl}}{} % add URL line breaks if available
\IfFileExists{bookmark.sty}{\usepackage{bookmark}}{\usepackage{hyperref}}
\hypersetup{
  pdftitle={PSY 364 Proposal},
  pdfauthor={Iris Zhong1 \& Yena Li1},
  pdflang={en-EN},
  pdfkeywords={keywords},
  hidelinks,
  pdfcreator={LaTeX via pandoc}}
\urlstyle{same} % disable monospaced font for URLs
\usepackage{graphicx}
\makeatletter
\def\maxwidth{\ifdim\Gin@nat@width>\linewidth\linewidth\else\Gin@nat@width\fi}
\def\maxheight{\ifdim\Gin@nat@height>\textheight\textheight\else\Gin@nat@height\fi}
\makeatother
% Scale images if necessary, so that they will not overflow the page
% margins by default, and it is still possible to overwrite the defaults
% using explicit options in \includegraphics[width, height, ...]{}
\setkeys{Gin}{width=\maxwidth,height=\maxheight,keepaspectratio}
% Set default figure placement to htbp
\makeatletter
\def\fps@figure{htbp}
\makeatother
\setlength{\emergencystretch}{3em} % prevent overfull lines
\providecommand{\tightlist}{%
  \setlength{\itemsep}{0pt}\setlength{\parskip}{0pt}}
\setcounter{secnumdepth}{-\maxdimen} % remove section numbering
% Make \paragraph and \subparagraph free-standing
\ifx\paragraph\undefined\else
  \let\oldparagraph\paragraph
  \renewcommand{\paragraph}[1]{\oldparagraph{#1}\mbox{}}
\fi
\ifx\subparagraph\undefined\else
  \let\oldsubparagraph\subparagraph
  \renewcommand{\subparagraph}[1]{\oldsubparagraph{#1}\mbox{}}
\fi
% Manuscript styling
\usepackage{upgreek}
\captionsetup{font=singlespacing,justification=justified}

% Table formatting
\usepackage{longtable}
\usepackage{lscape}
% \usepackage[counterclockwise]{rotating}   % Landscape page setup for large tables
\usepackage{multirow}		% Table styling
\usepackage{tabularx}		% Control Column width
\usepackage[flushleft]{threeparttable}	% Allows for three part tables with a specified notes section
\usepackage{threeparttablex}            % Lets threeparttable work with longtable

% Create new environments so endfloat can handle them
% \newenvironment{ltable}
%   {\begin{landscape}\begin{center}\begin{threeparttable}}
%   {\end{threeparttable}\end{center}\end{landscape}}
\newenvironment{lltable}{\begin{landscape}\begin{center}\begin{ThreePartTable}}{\end{ThreePartTable}\end{center}\end{landscape}}

% Enables adjusting longtable caption width to table width
% Solution found at http://golatex.de/longtable-mit-caption-so-breit-wie-die-tabelle-t15767.html
\makeatletter
\newcommand\LastLTentrywidth{1em}
\newlength\longtablewidth
\setlength{\longtablewidth}{1in}
\newcommand{\getlongtablewidth}{\begingroup \ifcsname LT@\roman{LT@tables}\endcsname \global\longtablewidth=0pt \renewcommand{\LT@entry}[2]{\global\advance\longtablewidth by ##2\relax\gdef\LastLTentrywidth{##2}}\@nameuse{LT@\roman{LT@tables}} \fi \endgroup}

% \setlength{\parindent}{0.5in}
% \setlength{\parskip}{0pt plus 0pt minus 0pt}

% \usepackage{etoolbox}
\makeatletter
\patchcmd{\HyOrg@maketitle}
  {\section{\normalfont\normalsize\abstractname}}
  {\section*{\normalfont\normalsize\abstractname}}
  {}{\typeout{Failed to patch abstract.}}
\patchcmd{\HyOrg@maketitle}
  {\section{\protect\normalfont{\@title}}}
  {\section*{\protect\normalfont{\@title}}}
  {}{\typeout{Failed to patch title.}}
\makeatother
\shorttitle{Proposal}
\keywords{keywords\newline\indent Word count: X}
\DeclareDelayedFloatFlavor{ThreePartTable}{table}
\DeclareDelayedFloatFlavor{lltable}{table}
\DeclareDelayedFloatFlavor*{longtable}{table}
\makeatletter
\renewcommand{\efloat@iwrite}[1]{\immediate\expandafter\protected@write\csname efloat@post#1\endcsname{}}
\makeatother
\usepackage{csquotes}
\ifxetex
  % Load polyglossia as late as possible: uses bidi with RTL langages (e.g. Hebrew, Arabic)
  \usepackage{polyglossia}
  \setmainlanguage[]{english}
\else
  \usepackage[shorthands=off,main=english]{babel}
\fi
\newlength{\cslhangindent}
\setlength{\cslhangindent}{1.5em}
\newenvironment{cslreferences}%
  {\setlength{\parindent}{0pt}%
  \everypar{\setlength{\hangindent}{\cslhangindent}}\ignorespaces}%
  {\par}

\title{PSY 364 Proposal}
\author{Iris Zhong\textsuperscript{1} \& Yena Li\textsuperscript{1}}
\date{}


\affiliation{\vspace{0.5cm}\textsuperscript{1} Smith College}

\begin{document}
\maketitle

\hypertarget{hypothesis}{%
\subsection{Hypothesis}\label{hypothesis}}

Due to the ongoing pandemic, many people have transitioned to work remotely. Therefore, their work-family relationship and chore division with their partner could be altered. Little research has been conducted on how working from home would affect romantic relationships. This study is trying to bridge the gap of this line of research, investigating how an individual or partner's telework could influence their perceived fairness in household chores, thereby affecting their relationship quality. We also incorporate the factor of gender into our research, exploring how males and females differ in their perceived fairness in chore division that leads to differences in relationship qualities.

Our research question is twofold:

\begin{enumerate}
\def\labelenumi{\arabic{enumi}.}
\item
  Whether the respondent or their partner is teleworking can affect their perceived fairness in household chores, and further influence their relationship quality;
\item
  Whether gender also plays a role in perceived chore fairness and consequently affects relationship quality.
\end{enumerate}

We plan to adopt an actor-partner model that is present in West, Pearson, Dovidio, Shelton, and Trail (2009). In this model, the outcome from the actor is influenced by both the actor themselves and their partner. Thus, we predict that whether the partner is teleworking can influence the actor's perceived fairness in household chores. Additionally, the partner's perceived fairness in chores can have an impact on the actor's relationship quality.

In addition, in Radcliffe and Cassell (2015)'s study, for heterosexual couples, when the female has more flexibility in working, she tends to take on more family responsibilities without noticing the unfairness. However, when the male has a more flexible job, the female still involves in the majority of family-related activities. Our research is built on top of this study. We hypothesize that telework allows for a more flexible working schedule, therefore creating more opportunities for the individual to participate in household tasks. On the other hand, females might report less perceived unfairness in chore division, regardless of their working flexibility, because of the traditional gender norms that women internalize.

Finally, we predict that perceived fairness in chore division could have a significant impact on romantic relationship quality. We believe that an even share of household tasks can lead to a higher level of relationship satisfaction.

\hypertarget{variables}{%
\subsection{Variables}\label{variables}}

We are planning to use the following variables:

\hypertarget{explanatory-variables}{%
\subsubsection{Explanatory Variables}\label{explanatory-variables}}

\begin{itemize}
\item
  \textbf{Gender} (\emph{gender}): whether the respondent is male or female
\item
  \textbf{Participant ID} (\emph{partID}): the ID of the participant
\item
  \textbf{Dyad ID} (\emph{dyadID}): the ID of the couple
\item
  \textbf{Telework} (\emph{Q139}): whether the respondent or their partner is/are teleworking
\item
  \textbf{Chore fairness} (\emph{fair\_chores}): how the respondent felt about the fairness of the division of household tasks on a particular day
\item
  \textbf{Time} (no direct variable available): indicate the entry in the daily diary
\end{itemize}

\hypertarget{response-variable}{%
\subsubsection{Response variable}\label{response-variable}}

\begin{itemize}
\tightlist
\item
  \textbf{Relationship quality} (\emph{qmi1} to \emph{qmi4}): the quality of marriage index (4 questions in total, 4 point scale)
\end{itemize}

\hypertarget{reading-list}{%
\subsection{Reading List}\label{reading-list}}

Below is the list of readings for annotated bibliography:

Oshio, Nozaki, and Kobayashi (2013)\\
Zhang, Moeckel, Moreno, Shuai, and Gao (2020)\\
Blom, Kraaykamp, and Verbakel (2017)\\
Faulkner, Davey, and Davey (2005)\\
Erchull, Liss, Axelson, Staebell, and Askari (2010)\\
Nourani, Seraj, Shakeri, and Mokhber (2019)\\
Dillaway and Broman (2001)\\
Norton (1983)\\
Song and Gao (2020)\\
Fleche, Lepinteur, and Powdthavee (2020)\\
Gillespie, Peterson, and Lever (2019)\\
Carriero and Todesco (2017)\\
Hu and Yucel (2018)\\
Jeffrey Hill et al. (2008)\\
Delanoeije, Verbruggen, and Germeys (2019)

We have used R (Version 4.0.3; R Core Team, 2020) and the R-package \emph{papaja} (Version 0.1.0.9997; Aust \& Barth, 2020) in this proposal.

\newpage

\hypertarget{references}{%
\section{References}\label{references}}

\begingroup
\setlength{\parindent}{-0.5in}
\setlength{\leftskip}{0.5in}

\hypertarget{refs}{}
\begin{cslreferences}
\leavevmode\hypertarget{ref-R-papaja}{}%
Aust, F., \& Barth, M. (2020). \emph{papaja: Create APA manuscripts with R Markdown}. Retrieved from \url{https://github.com/crsh/papaja}

\leavevmode\hypertarget{ref-blom2017couples}{}%
Blom, N., Kraaykamp, G., \& Verbakel, E. (2017). Couples' division of employment and household chores and relationship satisfaction: A test of the specialization and equity hypotheses. \emph{European Sociological Review}, \emph{33}(2), 195--208.

\leavevmode\hypertarget{ref-carriero2017interplay}{}%
Carriero, R., \& Todesco, L. (2017). The interplay between equity and gender ideology in perceived housework fairness: Evidence from an experimental vignette design. \emph{Sociological Inquiry}, \emph{87}(4), 561--585.

\leavevmode\hypertarget{ref-delanoeije2019boundary}{}%
Delanoeije, J., Verbruggen, M., \& Germeys, L. (2019). Boundary role transitions: A day-to-day approach to explain the effects of home-based telework on work-to-home conflict and home-to-work conflict. \emph{Human Relations}, \emph{72}(12), 1843--1868.

\leavevmode\hypertarget{ref-dillaway2001race}{}%
Dillaway, H., \& Broman, C. (2001). Race, class, and gender differences in marital satisfaction and divisions of household labor among dual-earner couples: A case for intersectional analysis. \emph{Journal of Family Issues}, \emph{22}(3), 309--327.

\leavevmode\hypertarget{ref-erchull2010well}{}%
Erchull, M. J., Liss, M., Axelson, S. J., Staebell, S. E., \& Askari, S. F. (2010). Well\ldots{} she wants it more: Perceptions of social norms about desires for marriage and children and anticipated chore participation. \emph{Psychology of Women Quarterly}, \emph{34}(2), 253--260.

\leavevmode\hypertarget{ref-faulkner2005gender}{}%
Faulkner, R. A., Davey, M., \& Davey, A. (2005). Gender-related predictors of change in marital satisfaction and marital conflict. \emph{The American Journal of Family Therapy}, \emph{33}(1), 61--83.

\leavevmode\hypertarget{ref-fleche2020gender}{}%
Fleche, S., Lepinteur, A., \& Powdthavee, N. (2020). Gender norms, fairness and relative working hours within households. \emph{Labour Economics}, \emph{65}, 101866.

\leavevmode\hypertarget{ref-gillespie2019gendered}{}%
Gillespie, B. J., Peterson, G., \& Lever, J. (2019). Gendered perceptions of fairness in housework and shared expenses: Implications for relationship satisfaction and sex frequency. \emph{Plos One}, \emph{14}(3), e0214204.

\leavevmode\hypertarget{ref-hu2018fairness}{}%
Hu, Y., \& Yucel, D. (2018). What fairness? Gendered division of housework and family life satisfaction across 30 countries. \emph{European Sociological Review}, \emph{34}(1), 92--105.

\leavevmode\hypertarget{ref-jeffrey2008exploring}{}%
Jeffrey Hill, E., Jacob, J. I., Shannon, L. L., Brennan, R. T., Blanchard, V. L., \& Martinengo, G. (2008). Exploring the relationship of workplace flexibility, gender, and life stage to family-to-work conflict, and stress and burnout. \emph{Community, Work and Family}, \emph{11}(2), 165--181.

\leavevmode\hypertarget{ref-norton1983measuring}{}%
Norton, R. (1983). Measuring marital quality: A critical look at the dependent variable. \emph{Journal of Marriage and the Family}, 141--151.

\leavevmode\hypertarget{ref-nourani2019relationship}{}%
Nourani, S., Seraj, F., Shakeri, M. T., \& Mokhber, N. (2019). The relationship between gender-role beliefs, household labor division and marital satisfaction in couples. \emph{Journal of Holistic Nursing and Midwifery}, \emph{29}(1), 43--49.

\leavevmode\hypertarget{ref-oshio2013division}{}%
Oshio, T., Nozaki, K., \& Kobayashi, M. (2013). Division of household labor and marital satisfaction in china, japan, and korea. \emph{Journal of Family and Economic Issues}, \emph{34}(2), 211--223.

\leavevmode\hypertarget{ref-radcliffe2015flexible}{}%
Radcliffe, L. S., \& Cassell, C. (2015). Flexible working, work--family conflict, and maternal gatekeeping: The daily experiences of dual-earner couples. \emph{Journal of Occupational and Organizational Psychology}, \emph{88}(4), 835--855.

\leavevmode\hypertarget{ref-R-base}{}%
R Core Team. (2020). \emph{R: A language and environment for statistical computing}. Vienna, Austria: R Foundation for Statistical Computing. Retrieved from \url{https://www.R-project.org/}

\leavevmode\hypertarget{ref-song2020does}{}%
Song, Y., \& Gao, J. (2020). Does telework stress employees out? A study on working at home and subjective well-being for wage/salary workers. \emph{Journal of Happiness Studies}, \emph{21}(7), 2649--2668.

\leavevmode\hypertarget{ref-west2009superordinate}{}%
West, T. V., Pearson, A. R., Dovidio, J. F., Shelton, J. N., \& Trail, T. E. (2009). Superordinate identity and intergroup roommate friendship development. \emph{Journal of Experimental Social Psychology}, \emph{45}(6), 1266--1272.

\leavevmode\hypertarget{ref-zhang2020work}{}%
Zhang, S., Moeckel, R., Moreno, A. T., Shuai, B., \& Gao, J. (2020). A work-life conflict perspective on telework. \emph{Transportation Research Part A: Policy and Practice}, \emph{141}, 51--68.
\end{cslreferences}

\endgroup


\end{document}
