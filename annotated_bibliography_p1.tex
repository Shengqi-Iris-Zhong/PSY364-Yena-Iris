% Options for packages loaded elsewhere
\PassOptionsToPackage{unicode}{hyperref}
\PassOptionsToPackage{hyphens}{url}
%
\documentclass[
  english,
  man]{apa6}
\usepackage{lmodern}
\usepackage{amssymb,amsmath}
\usepackage{ifxetex,ifluatex}
\ifnum 0\ifxetex 1\fi\ifluatex 1\fi=0 % if pdftex
  \usepackage[T1]{fontenc}
  \usepackage[utf8]{inputenc}
  \usepackage{textcomp} % provide euro and other symbols
\else % if luatex or xetex
  \usepackage{unicode-math}
  \defaultfontfeatures{Scale=MatchLowercase}
  \defaultfontfeatures[\rmfamily]{Ligatures=TeX,Scale=1}
\fi
% Use upquote if available, for straight quotes in verbatim environments
\IfFileExists{upquote.sty}{\usepackage{upquote}}{}
\IfFileExists{microtype.sty}{% use microtype if available
  \usepackage[]{microtype}
  \UseMicrotypeSet[protrusion]{basicmath} % disable protrusion for tt fonts
}{}
\makeatletter
\@ifundefined{KOMAClassName}{% if non-KOMA class
  \IfFileExists{parskip.sty}{%
    \usepackage{parskip}
  }{% else
    \setlength{\parindent}{0pt}
    \setlength{\parskip}{6pt plus 2pt minus 1pt}}
}{% if KOMA class
  \KOMAoptions{parskip=half}}
\makeatother
\usepackage{xcolor}
\IfFileExists{xurl.sty}{\usepackage{xurl}}{} % add URL line breaks if available
\IfFileExists{bookmark.sty}{\usepackage{bookmark}}{\usepackage{hyperref}}
\hypersetup{
  pdftitle={PSY 364 - Annotated Bibliography Part 1},
  pdfauthor={Iris Zhong1},
  pdflang={en-EN},
  pdfkeywords={keywords},
  hidelinks,
  pdfcreator={LaTeX via pandoc}}
\urlstyle{same} % disable monospaced font for URLs
\usepackage{graphicx}
\makeatletter
\def\maxwidth{\ifdim\Gin@nat@width>\linewidth\linewidth\else\Gin@nat@width\fi}
\def\maxheight{\ifdim\Gin@nat@height>\textheight\textheight\else\Gin@nat@height\fi}
\makeatother
% Scale images if necessary, so that they will not overflow the page
% margins by default, and it is still possible to overwrite the defaults
% using explicit options in \includegraphics[width, height, ...]{}
\setkeys{Gin}{width=\maxwidth,height=\maxheight,keepaspectratio}
% Set default figure placement to htbp
\makeatletter
\def\fps@figure{htbp}
\makeatother
\setlength{\emergencystretch}{3em} % prevent overfull lines
\providecommand{\tightlist}{%
  \setlength{\itemsep}{0pt}\setlength{\parskip}{0pt}}
\setcounter{secnumdepth}{-\maxdimen} % remove section numbering
% Make \paragraph and \subparagraph free-standing
\ifx\paragraph\undefined\else
  \let\oldparagraph\paragraph
  \renewcommand{\paragraph}[1]{\oldparagraph{#1}\mbox{}}
\fi
\ifx\subparagraph\undefined\else
  \let\oldsubparagraph\subparagraph
  \renewcommand{\subparagraph}[1]{\oldsubparagraph{#1}\mbox{}}
\fi
% Manuscript styling
\usepackage{upgreek}
\captionsetup{font=singlespacing,justification=justified}

% Table formatting
\usepackage{longtable}
\usepackage{lscape}
% \usepackage[counterclockwise]{rotating}   % Landscape page setup for large tables
\usepackage{multirow}		% Table styling
\usepackage{tabularx}		% Control Column width
\usepackage[flushleft]{threeparttable}	% Allows for three part tables with a specified notes section
\usepackage{threeparttablex}            % Lets threeparttable work with longtable

% Create new environments so endfloat can handle them
% \newenvironment{ltable}
%   {\begin{landscape}\begin{center}\begin{threeparttable}}
%   {\end{threeparttable}\end{center}\end{landscape}}
\newenvironment{lltable}{\begin{landscape}\begin{center}\begin{ThreePartTable}}{\end{ThreePartTable}\end{center}\end{landscape}}

% Enables adjusting longtable caption width to table width
% Solution found at http://golatex.de/longtable-mit-caption-so-breit-wie-die-tabelle-t15767.html
\makeatletter
\newcommand\LastLTentrywidth{1em}
\newlength\longtablewidth
\setlength{\longtablewidth}{1in}
\newcommand{\getlongtablewidth}{\begingroup \ifcsname LT@\roman{LT@tables}\endcsname \global\longtablewidth=0pt \renewcommand{\LT@entry}[2]{\global\advance\longtablewidth by ##2\relax\gdef\LastLTentrywidth{##2}}\@nameuse{LT@\roman{LT@tables}} \fi \endgroup}

% \setlength{\parindent}{0.5in}
% \setlength{\parskip}{0pt plus 0pt minus 0pt}

% \usepackage{etoolbox}
\makeatletter
\patchcmd{\HyOrg@maketitle}
  {\section{\normalfont\normalsize\abstractname}}
  {\section*{\normalfont\normalsize\abstractname}}
  {}{\typeout{Failed to patch abstract.}}
\patchcmd{\HyOrg@maketitle}
  {\section{\protect\normalfont{\@title}}}
  {\section*{\protect\normalfont{\@title}}}
  {}{\typeout{Failed to patch title.}}
\makeatother
\shorttitle{Annotated Bibliography}
\keywords{keywords\newline\indent Word count: X}
\DeclareDelayedFloatFlavor{ThreePartTable}{table}
\DeclareDelayedFloatFlavor{lltable}{table}
\DeclareDelayedFloatFlavor*{longtable}{table}
\makeatletter
\renewcommand{\efloat@iwrite}[1]{\immediate\expandafter\protected@write\csname efloat@post#1\endcsname{}}
\makeatother
\usepackage{csquotes}
\ifxetex
  % Load polyglossia as late as possible: uses bidi with RTL langages (e.g. Hebrew, Arabic)
  \usepackage{polyglossia}
  \setmainlanguage[]{english}
\else
  \usepackage[shorthands=off,main=english]{babel}
\fi
\newlength{\cslhangindent}
\setlength{\cslhangindent}{1.5em}
\newenvironment{cslreferences}%
  {\setlength{\parindent}{0pt}%
  \everypar{\setlength{\hangindent}{\cslhangindent}}\ignorespaces}%
  {\par}

\title{PSY 364 - Annotated Bibliography Part 1}
\author{Iris Zhong\textsuperscript{1}}
\date{}


\affiliation{\vspace{0.5cm}\textsuperscript{1} Smith College}

\begin{document}
\maketitle

\hypertarget{what-fairness-gendered-division-of-housework-and-family-life-satisfaction-across-30-countries}{%
\paragraph{What Fairness? Gendered Division of Housework and Family Life Satisfaction across 30 Countries}\label{what-fairness-gendered-division-of-housework-and-family-life-satisfaction-across-30-countries}}

Hu and Yucel (2018) conduct a cross-cultural study that examines how inter-gender relational fairness comparison between partners, intra-gender referential comparison within country (how equal a woman's share of housework is compared to other females from the same country), and their interaction influence women's family life satisfaction. The study uses the 2012 survey ``Family and Changing Gender Roles'' survey from International Social Survey Programme and only focuses on female respondents with heterosexual cohabiting partners. The final sample consists of 10467 women from 30 countries. The research adopts a mixed effect multilevel model, where level 1 is individual measures, and level 2 is country measures. The results demonstrate that women report a higher level of family life satisfaction when they perceive the housework division between the partners to be fair rather than unfair; women enjoy greater family life satisfaction when their share of housework is more even than the other women with similar gender ideology in the same country; relational fairness has a greater impact on women's family life satisfaction when their share of housework is considered less egalitarian. The important takeaway from this research is that perceived fairness can be twofold: comparison with the partner, and comparison with female counterparts in the same social context. Since we haven't included the latter factor, we may have missed the influence of social norms in housework division on an individual's satisfaction. Regardless, we could predict that in our study, women report greater family life satisfaction when the housework division between the partners is more even. Regarding the fairness variable, the authors also concern that female who report taking less than their fair share of housework may also have experienced lower satisfaction because of inequality in housework division. They show that this possibility is not present in the sample, as a robust test excluding these female results in similar conclusions. We could use the same method to explore this variable. It also offers a list of covariates worth considering, such as education, marital status, socioeconomic status, and family income.

\hypertarget{the-interplay-between-equity-and-gender-ideology-in-perceived-housework-fairness-evidence-from-an-experimental-vignette-design}{%
\paragraph{The Interplay between Equity and Gender Ideology in Perceived Housework Fairness: Evidence from an Experimental Vignette Design}\label{the-interplay-between-equity-and-gender-ideology-in-perceived-housework-fairness-evidence-from-an-experimental-vignette-design}}

Carriero and Todesco (2017) examine how equity theory and gender theory both play a role in perceived housework fairness. Equity theory holds that for a person who contributes more to household income or has longer working hours, it is considered fair even if they do less housework. Gender theory denotes whether a person embraces a more traditional or egalitarian gender ideology. The study sampled dual-earner married or cohabitating couples with at least one child under 13 years old from Italy. There were 1656 individuals in the initial phone call interview during 2010 and 2011, in which they were asked about the division of domestic and care tasks, perceived fairness, gender roles, and paid work. Then, 770 individuals participated in the online vignette questionnaire in 2013. In the online survey, respondents read two vignette stories, one related to the fairness of a given family arrangement in terms of paid and unpaid work, the other on the justifiability of a request to renegotiate housework division. Respondents were asked to rate on a scale (100-point and 10-point, respectively) to indicate whether the housework share is fair, or whether the request to renegotiate housework division is justifiable. In these stories, factors such as whether the vignette has children and vignette's contribution to family income are randomized for each participant. The results show that when the female vignette works longer and does less share of housework than her husband, the chore division is considered approximately fair as well. Gender ideology influences perceived fairness, such that individuals with an egalitarian attitude tend to judge inequalities in housework division that are unfavorable to the wife more negatively than traditional-minded respondents. Our proposed research is related to this paper, since telework can potentially change the working hours, and subsequently influence perceived fairness in chore division. This paper also reminds us that controlling over variables such as working hours and gender ideology is critical, and we will add these variables as covariates if possible.

\hypertarget{does-telework-stress-employees-out-a-study-on-working-at-home-and-subjective-well-being-for-wagesalary-workers}{%
\paragraph{Does Telework Stress Employees Out? A Study on Working at Home and Subjective Well-Being for Wage/Salary Workers}\label{does-telework-stress-employees-out-a-study-on-working-at-home-and-subjective-well-being-for-wagesalary-workers}}

Song and Gao (2020)'s study investigates if working at home improves instantaneous subjective well-being for wage or salary workers. In particular, the paper separates bringing work home from teleworking, because these two types of working from home have different connotations. The study utilizes data from the American Time Use Survey in 2010, 2012, and 2013. The survey was conducted through telephone interviewing and collected a detailed log of the participant's activities during a 24-h period on one day of the week. It was also accompanied by a supplemental survey asking about the respondent's feelings about three random activities on that day. The survey produced 11,793 episodes of activities from 3962 respondents. An individual fixed effect model shows that workers tend to experience a higher level of stress when teleworking than working in the workplace. However, the effect of working from home is heterogeneous across gender and parental status, and varies on weekdays versus weekends. For instance, teleworking on weekends or holidays is less painful than working in the workplace for mothers. This research challenges the commonsense that teleworking brings more flexibility and should reduce stress for workers. We can predict that a bigger picture in our study can be that teleworking could reduce relationship satisfaction due to increased stress. Though the authors have not investigated the explanations of teleworking stress, it is likely that work-family conflict, such as chore division, can be one of them. Additionally, the effect of working from home on weekdays is different from that on weekends or holidays. Therefore, when dealing with the daily diary data, we need to incorporate the day of the week as a control variable.

\hypertarget{gender-norms-fairness-and-relative-working-hours-within-households}{%
\paragraph{Gender norms, fairness and relative working hours within households}\label{gender-norms-fairness-and-relative-working-hours-within-households}}

Fleche, Lepinteur, and Powdthavee (2020) investigate how working hours and gender ideology could influence an individual's satisfaction. The study uses existing data from the US (ATUS, 2012-2013 and PSID, 2015-2016), the UK (BHPS, longitudinal from 1996 to 2008), and Germany (SOEP, longitudinal from 1995 to 2012). The authors only select married and cohabiting individuals aged between 20 and 60 years old, resulting in a sample size of 34,362. All datasets include respondent's and their partner's working patterns, and respondent's own satisfaction. Regression and fixed effect models are used. Results demonstrate that women who work longer hours than their husbands report significantly lower life satisfaction. In addition, for women who work longer than their husbands, the time they spend on housework is not less than women who work less or the same as their spouses. Finally, only females who hold a more egalitarian ideology and who work more than their husbands perceive housework division to be unfair. The study illustrates how gender and working hours interplay to influence satisfaction. If we add gender ideology and working time as a factor in our study, we could predict that there is an interaction between gender, working hours and ideology on perceived fairness, such that females who work longer with an egalitarian view tend to believe the housework division is unfair.

\hfill\break
\hfill\break

We have used R (Version 4.0.3; R Core Team, 2020) and the R-package \emph{papaja} (Version 0.1.0.9997; Aust \& Barth, 2020) in this proposal.

\newpage

\hypertarget{references}{%
\section{References}\label{references}}

\begingroup
\setlength{\parindent}{-0.5in}
\setlength{\leftskip}{0.5in}

\hypertarget{refs}{}
\begin{cslreferences}
\leavevmode\hypertarget{ref-R-papaja}{}%
Aust, F., \& Barth, M. (2020). \emph{papaja: Create APA manuscripts with R Markdown}. Retrieved from \url{https://github.com/crsh/papaja}

\leavevmode\hypertarget{ref-carriero2017interplay}{}%
Carriero, R., \& Todesco, L. (2017). The interplay between equity and gender ideology in perceived housework fairness: Evidence from an experimental vignette design. \emph{Sociological Inquiry}, \emph{87}(4), 561--585.

\leavevmode\hypertarget{ref-fleche2020gender}{}%
Fleche, S., Lepinteur, A., \& Powdthavee, N. (2020). Gender norms, fairness and relative working hours within households. \emph{Labour Economics}, \emph{65}, 101866.

\leavevmode\hypertarget{ref-hu2018fairness}{}%
Hu, Y., \& Yucel, D. (2018). What fairness? Gendered division of housework and family life satisfaction across 30 countries. \emph{European Sociological Review}, \emph{34}(1), 92--105.

\leavevmode\hypertarget{ref-R-base}{}%
R Core Team. (2020). \emph{R: A language and environment for statistical computing}. Vienna, Austria: R Foundation for Statistical Computing. Retrieved from \url{https://www.R-project.org/}

\leavevmode\hypertarget{ref-song2020does}{}%
Song, Y., \& Gao, J. (2020). Does telework stress employees out? A study on working at home and subjective well-being for wage/salary workers. \emph{Journal of Happiness Studies}, \emph{21}(7), 2649--2668.
\end{cslreferences}

\endgroup


\end{document}
